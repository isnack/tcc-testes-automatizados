

\chapter{CONSIDERAÇÕES FINAIS} 
\begin{comment}
\par A conclusão deste trabalho é \ldots

\par Assim conclui-se que \ldots
\end{comment}


\par Com a realização desta pesquisa, foi criado um sistema de folha de pagamento \textit{web} utilizando testes automatizados, foram descritos o funcionamento dos testes, a preparação de um ambiente de desenvolvimento e a demonstração do funcionamento dos testes automatizados, e com isso, se cria um \textit{software} com um índice menor de falhas.

\par Foi utilizado em conjunto com os testes automatizados, a integração contínua, que visa a cada nova melhoria no código estar verificando se isso não vai ocasionar nenhuma falha ou até mesmo quebrar a \textit{build}, com isso o desenvolvedor garante cada vez mais entregar no final um \textit{software} de qualidade e com menos \textit{bugs}.

\par E também, em conjunto aos testes foi utilizado o relatório de cobertura cujo seu principal objetivo é alertar os desenvolvedores sobre a porcentagem que o código está coberto, como é possível verificar qual parte do código necessita de mais testes, para aumentar esse índice de cobertura.

\par O principal objetivo desse trabalho foi demonstrar que a utilização de testes no desenvolvimento de software contribui para o aumento da qualidade da aplicação a ser desenvolvida, prevenindo erros que possam ocorrer em uma aplicação.

\par Por fim, pode-se afirmar que o presente trabalho realizou seus  objetivos,os quais eram: desenvolver uma folha de pagamento \textit{web}, na qual iria proporcionar a realização dos testes automatizados que foram propostos por este trabalho, proporcionando um maior conhecimento na utilização de testes automatizados de \textit{softwares} e seus mais variados tipos, para começar a desenvolver utilizando a prática de desenvolvimento de software TDD (desenvolvimento orientado a teste) que requer um esforço maior, porque é uma prática diferente do que a maioria dos desenvolvedores estão acostumados, em meio essa dificuldade, percebeu-se que com a utilização dos testes nessa prática de desenvolvimento, consegue-se evitar erros no final do projeto, pois foram realizados testes em todo o processo de desenvolvimento do sistema.

\par Ainda, esta pesquisa foi de grande relevância aos participantes do projeto, pois contribui para uma ampla visão sobre os benefícios de se utilizar testes automatizados em todo o desenvolvimento do \textit{software} e um conhecimento vasto nas tecnologias utilizadas.

\par Como futuros trabalhos sugerimos a utilização do BDD (Behavior Driven Development) e das ferramentas para análise na qualidade do código, pois este apresenta uma abordagem diferenciada, interessante e com mais vantagens ainda em sua utilização.


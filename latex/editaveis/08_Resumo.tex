






% --- resumo em português ---

\begin{OnehalfSpacing} 

\noindent \imprimirAutorCitacaoMaiuscula. {\bfseries\imprimirtitulo}. {\imprimirdata}.  Monografia -- Curso de {\MakeUppercase\imprimircurso}, {\imprimirinstituicao}, {\imprimirlocal}, {\imprimirdata}.

\vspace{\onelineskip}
\vspace{\onelineskip}
\vspace{\onelineskip}
\vspace{\onelineskip}

\begin{resumo}
~\\
%início do texto do resumo
\begin{comment}


\noindent Este trabalho apresenta \ldots
 \end{comment}
 
 
 \par Atualmente há uma grande demanda de desenvolvimento de \textit{software}, e com isso o usuário final exige um sistema com o menor número de \textit{bugs} possíveis. Neste contexto, foi feita a demonstração e utilização dos testes automatizados de \textit{software}, por meio da prática de desenvolvimento TDD, que mostra como os testes ajudam a entregar um sistema de qualidade e com uma taxa mínima de erros. Durante a pesquisa foi desenvolvida uma folha de pagamento \textit{web} e através dela foram realizados todos os testes propostos para esta aplicação. Foi utilizada em conjunto com os testes automatizados, a integração contínua, que visa a cada nova melhoria no sistema verificar se os testes irão passar novamente ou se irão quebrar a \textit{build}, e o relatório de cobertura, que auxilia o desenvolvedor a visualizar qual é o índice de cobertura do código e qual parte do sistema necessita de mais testes. A utilização dessas ferramentas só agregam e ajudam no processo de desenvolvimento dos \textit{softwares}, pois possui \textit{fast feedback} (retorno rápido) para o desenvolvedor, que facilita na detecção e solução de problemas e visa produzir um sistema com o menor índice de erros. Este trabalho foi desenvolvido por meio de pesquisa aplicada, visando otimizar o resultado final, com o objetivo de resolver problemas específicos identificados durante todo o desenvolvimento dos testes.
 
 
 
 \begin{comment}
 

 Este trabalho enquadra-se no tipo de pesquisa aplicada, pois foi desenvolvido um sistema com o objetivo de resolver problemas específicos. 
 

 
 
 
 
Palavras-chave:
 \end{comment}
%fim do texto do resumo
\vspace{\onelineskip}
\vspace*{\fill}
\noindent \textbf{Palavras-chave}: \imprimirPalavraChaveUm. \imprimirPalavraChaveDois. \imprimirPalavraChaveTres.
\vspace{\onelineskip}
\end{resumo}

\end{OnehalfSpacing}


